 % Please do not change the document class
\documentclass{scrartcl}

% Please do not change these packages
\usepackage[hidelinks]{hyperref}
\usepackage[none]{hyphenat}
\usepackage{setspace}
\usepackage{graphicx}
\graphicspath{ {Images/} }
\doublespace

% You may add additional packages here
\usepackage{amsmath}

% Please include a clear, concise, and descriptive title
\title{Clothes or Body Type}

% Please do not change the subtitle
\subtitle{COMP230 - Ethics Essay}

% Please put your student number in the author field
\author{1506530}

\begin{document}

\maketitle

\abstract {In mainstream game development, Objectification of the human body is a critical issue, though the exact cause this is uncertain. Three causes for this could be Body Type, Clothing and Pose. But which one is the biggest culprit for objectification? Taking the three variables into account we split them further into three levels having two extremes and in-between e.g. Almost no clothes, fully clothes and some clothes.
		\newpage
	Goal:  To determine weather or not in video games the Body Type or Amount of clothes worn effect a player’s perception of objectification on the games avatar.
		\newline
	Hypothesis: Lack of clothes is not the primarily contributor to the objectification of an avatar.
		\newline
	Hypothesis: Having a hyper sexualized body type may be the primary contributor.
		\newline
	Method: I shall be using a survey based primary research to evaluate peoples opinions on this subject. They will be closed ended questions with limited choice in answer i.e. a selection of three images will be presented to them and they must select one. These three examples will be one of the three variables used.
	\newline
	These results will be used to compare the views of men and women when choosing their avatar.}
	
\newpage

\section{Introduction}

According to Entertainment Software Association(ESA), in the USA of gaming females take up almost 48\% of the gaming community \cite{SexistGames}, so it is important to always take into consideration what they think of the avatars they have been given. Evidence suggests that a lot of designers do not take this into consideration\cite{VideoGameMedia}. This is unfortunate as this approach can push away the female players and tends to lead to the sexist view that many people believe the gaming industry has. Though this is an issue that should be looked at what this survey intends on finding out is weather the female avatars "Sexist" look is based on the hypersexualized body type or the clothes that they wear.
		\newline
The Hypothesis for this research is that the lack of clothes is not the primarily contributor to the objectification or "sexualisation" of an avatar. But rather having a hyper sexualized body type may be the primary contributor. I shall be using a survey based primary research to evaluate peoples views on this. They will be closed ended questions with limited choice in answer. These examples will be following defined variables, and the results will be split into male/female base and compared for their view on sexualized avatars.

\section{Background}

Over the recent years many papers have been looking heavily into the gender divide. They tend to point out the stereotypical female character and how they should be changed\cite{LaraPhenomenon,MirrorSexist,GenderSteriotype}. As well as this there had been discussion on this topic in the media\cite{SexualisationSTOP,LetsTalk}. Though this does not point out the underlying issue, what is it that the gaming populous see as the cause for these conceptions? In the paper Hypersexual Avatars - Who wants them?\cite{WhowantsThem} they look into what avatars people chose when confronted with a variety of stereotyped females and females that have not been labeled as stereotyped. They also looked into hypersexualized bodies and clothing levels, though this was not looked into as much as the stereotyping. This is unfortunate as looking into how people perceive female avatars from appearance alone would have given a clearer answer on the conception of female objectification. 

\section{Method}

\subsection{Definitions}

The variables that this paper will be using to judge players perception of the female avatar will be hypersexualized and "average" bodies and the level of clothes worn for both instances of body type.
\newline
In the book Gender Inclusive Game Design\cite{Graner}, Graner describes a hypersexual body as having: unnaturally large breasts, a shapely behind, a narrow waist. In addition to this in the book Provocateur: Images for Women and Minorities in Advertising\cite{WomenAds}, young features, exaggerated leg length and visible cleavage between the buttocks and breast will also be used as standards for hypersexualisation.

The level of clothing used will be determined by the article Disco Clothing, Female Sexual Motivation and Relationship Status: Is She Dressed to Impress\cite{SkinExpose}. This sexology research indicates that women with more visible skin are typically more sexually available. For this paper clothing will be defined by more body skin shown and less to no body skin shown.

\subsection{Survey}
This paper will be using a survey based primary research to indicate the views of people on the avatars they are given to choose from. When choosing an avatar they will be forced to choose between two instances of avatars and example follows:

This information will be used to discover the differences or similarities between different players choices in avatars. We will be analyzing the different choices that people make on weather the body type or clothing level effects the conception of their avatar.

\section{Results}
Looking at the results in the appendix and calculating the total number of votes that each category received. Then converting them into a percentage of total votes received across the survey, gives a reasonably accurate representation of what people thought. With 37\% of people voting for a non-hypersexualized, clothed character and 32\% voting for a hypersexualized, clothed character there seems to be little difference in preference when characters are clothed. When removing the clothing the results seem to remain similar with 17\% votes for a non-hypersexualized, less clothed character and 13\% for a hypersexualized, less clothed character.

\section{Discussion}
On their own these results would suggest that there is little difference in the preferences of players to what their character look like. Though when compared looking at the survey results players had a tendency to choose the non-hypersexualized character with less clothes when presented with a hypersexualized character with either less clothes or fully clothed. This supports that the hypothesis of this paper was correct, that players will choose a non-hypersexualized character over a hypersexualized character whether they are clothed or not.

\section{Conclusion}
Though the results indicated towards players do not have much or a preference difference when selecting an avatar, further research is needed. A larger sample size would give a much clearer idea on what players thought as well as creating avatars specifically for the survey rather than using preexisting ones. Another approach would be to view the male sexualisation in games and perform a survey for that subject, also to get a more accurate view splitting the results to know what the gender divide would be on this subject.



\bibliographystyle{ieeetr}
\bibliography{Ethics-Bib}

\end{document}
